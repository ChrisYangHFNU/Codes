\documentclass{beamer}
\usepackage[UTF8,noindent]{ctexcap}

\begin{document}
\begin{frame}{古中国数学}{定理发现}
中国在3000多年前就知道勾股数的概念,比古希腊更早一些。

《周髀算经》的记载:
\begin{itemize}
\item 公元前 11 世纪,商高答周公问:
\begin{quote}
勾广三,股修四,径隅五。
\end{quote}
\item 又载公元前 7--6 世纪陈子答荣方问,表述了勾股定理的一般形式:
\begin{quote}
若求邪至日者,以日下为勾,日高为股,勾股各自乘,并而开方除之,得邪至日。
\end{quote}
\end{itemize}
\end{frame}

\end{document}