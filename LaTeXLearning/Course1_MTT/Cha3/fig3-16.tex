\documentclass{standalone}
\usepackage{tikz}
\usetikzlibrary{calc,intersections,through}
\usepackage{circuitikz}
\usepackage{siunitx}

\makeatletter
\pgfcircdeclarebipole{}{\ctikzvalof{bipoles/vsourcesin/height}}{sVpm}{\ctikzvalof{bipoles/vsourcesin/height}}{\ctikzvalof{bipoles/vsourcesin/width}}{

  \pgfsetlinewidth{\pgfkeysvalueof{/tikz/circuitikz/bipoles/thickness}\pgfstartlinewidth}
  \pgfpathellipse{\pgfpointorigin}{\pgfpoint{0}{\pgf@circ@res@up}}{\pgfpoint{\pgf@circ@res@left}{0}}
  \pgfusepath{draw}     
      \pgftext[bottom,rotate=90,y=\ctikzvalof{bipoles/vsourceam/margin}\pgf@circ@res@down]{\scriptsize$-$}
      \pgftext[top,rotate=90,y=\ctikzvalof{bipoles/vsourceam/margin}\pgf@circ@res@up]{\scriptsize$+$}

    \pgf@circ@res@up = .35\pgf@circ@res@up
    \pgfscope
      \pgftransformrotate{90}
      \pgfpathmoveto{\pgfpoint{-\pgf@circ@res@up}{0cm}}
      \pgfpathsine{\pgfpoint{.5\pgf@circ@res@up}{.4\pgf@circ@res@up}}
      \pgfpathcosine{\pgfpoint{.5\pgf@circ@res@up}{-.4\pgf@circ@res@up}}
      \pgfpathsine{\pgfpoint{.5\pgf@circ@res@up}{-.4\pgf@circ@res@up}}
      \pgfpathcosine{\pgfpoint{.5\pgf@circ@res@up}{.4\pgf@circ@res@up}}
      \pgfusepath{draw}
    \endpgfscope
}
\def\pgf@circ@sVpm@path#1{\pgf@circ@bipole@path{sVpm}{#1}}
\compattikzset{sinusoidal voltage source pm/.style = {\circuitikzbasekey, /tikz/to path=\pgf@circ@sVpm@path, \circuitikzbasekey/bipole/is voltage=true, \circuitikzbasekey/bipole/is voltageoutsideofsymbol=true, v=#1 }}
\compattikzset{sVpm/.style = {\comnpatname sinusoidal voltage source pm = #1}}

\pgfcircdeclarebipole{}{\ctikzvalof{bipoles/vsourcesin/height}}{sVpmh}{\ctikzvalof{bipoles/vsourcesin/height}}{\ctikzvalof{bipoles/vsourcesin/width}}{

  \pgfsetlinewidth{\pgfkeysvalueof{/tikz/circuitikz/bipoles/thickness}\pgfstartlinewidth}
  \pgfpathellipse{\pgfpointorigin}{\pgfpoint{0}{\pgf@circ@res@up}}{\pgfpoint{\pgf@circ@res@left}{0}}
  \pgfusepath{draw}     
      \pgftext[center,x=0.2\pgf@circ@res@down-\ctikzvalof{bipoles/vsourceam/margin}\pgf@circ@res@down]{\scriptsize$-$}
      \pgftext[top,rotate=90,y=\ctikzvalof{bipoles/vsourceam/margin}\pgf@circ@res@up]{\scriptsize$+$}

    \pgf@circ@res@up = .35\pgf@circ@res@up
    \pgfscope
      \pgftransformrotate{90}
      \pgfpathmoveto{\pgfpoint{-\pgf@circ@res@up}{0cm}}
      \pgfpathsine{\pgfpoint{.5\pgf@circ@res@up}{.4\pgf@circ@res@up}}
      \pgfpathcosine{\pgfpoint{.5\pgf@circ@res@up}{-.4\pgf@circ@res@up}}
      \pgfpathsine{\pgfpoint{.5\pgf@circ@res@up}{-.4\pgf@circ@res@up}}
      \pgfpathcosine{\pgfpoint{.5\pgf@circ@res@up}{.4\pgf@circ@res@up}}
      \pgfusepath{draw}
    \endpgfscope
}
\def\pgf@circ@sVpmh@path#1{\pgf@circ@bipole@path{sVpmh}{#1}}
\compattikzset{sinusoidal voltage source pmh/.style = {\circuitikzbasekey, /tikz/to path=\pgf@circ@sVpmh@path, \circuitikzbasekey/bipole/is voltage=true, \circuitikzbasekey/bipole/is voltageoutsideofsymbol=true, v=#1 }}
\compattikzset{sVpmh/.style = {\comnpatname sinusoidal voltage source pmh = #1}}
\makeatother


\begin{document}

\begin{circuitikz}
    \draw (-2,2)
    to[sVpm] (-2,-2)
    to[short] (2,-2);
    \draw (2,-2.2) -- (2,-1.8) -- (6,-1.8) -- (6,-2.2) -- cycle;
    \draw[short] (6,-2)
    to[short] (8,-2)
    to[R=$Z_L$] (8,2)
    to[short] (8,2)
    to[short] (6,2);
    \draw (6,1.8) -- (6,2.2) -- (2,2.2) -- (2,1.8) -- cycle;
    \draw[short] (2,2)
    to[short] (1,2)
    to[R=$Z_s$] (-1,2)
    to[short] (-2,2);

    \draw (4,-0.5)node[label={[font=\large]above:$Z_0=50\Omega$}] {};
    \draw (3.5,2)node[label={[font=\footnotesize]right:$l$}] {};
    \draw (-2.25,0)node[label={[font=\small]left:$V_s$}] {};

    \draw[->] (2,1)node[label={[font=\footnotesize]right:$\Gamma_S$}]{} -- (2,1.5) -- (1.8,1.5);
    \draw[->] (6,1)node[label={[font=\footnotesize]left:$\Gamma_L$}]{} -- (6,1.5) -- (6.2,1.5);
    \draw (2,1) node[label={[font=\footnotesize]left:$+$}]{};
    \draw (2,-1) node[label={[font=\footnotesize]left:$-$}]{};
    \draw (2,0) node[label={[font=\footnotesize]left:$V_i$}]{};
    \draw (6,1) node[label={[font=\footnotesize]right:$+$}]{};
    \draw (6,-1) node[label={[font=\footnotesize]right:$-$}]{};
    \draw (6,0) node[label={[font=\footnotesize]right:$V_L$}]{};

    \draw[->] (1.5,-2.5) -- (8.5,-2.5)node[label={[font=\footnotesize]below:$z$}]{};
    \draw (2,-2.5) node[label={[font=\footnotesize]below:$0$}]{};
    \draw (6,-2.5) node[label={[font=\footnotesize]below:$l$}]{};
    \foreach \x in {2,6}
    \draw (\x,-2.6) -- (\x,-2.4);

    \draw[->] (6.5,-3.5) -- (0,-3.5)node[label={[font=\footnotesize]below:$d$}]{};
    \draw (6,-3.5) node[label={[font=\footnotesize]below:$0$}]{};
    \draw (2,-3.5) node[label={[font=\footnotesize]below:$l$}]{};
    \foreach \x in {2,6}
    \draw (\x,-3.6) -- (\x,-3.4);

\end{circuitikz}

\end{document}